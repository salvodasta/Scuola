\documentclass{beamer}

\usepackage[utf8]{inputenc}
\usepackage[italian]{babel}
\usepackage{tikz}
\usetheme{AnnArbor}
\setbeamercovered{dynamic}
\usepackage{default}
 \usepackage{cancel}
\title{RSA (Crittografia Asimmetrica)}
\author{Prof. Salvatore D'Asta}
\begin{document}
\begin{frame}
\maketitle
\end{frame}

\begin{frame}
\frametitle{Algoritmo RSA}
Alice vuole consentire a Bob di inviarle un messaggio $ m $ segreto.
\begin{enumerate}
	\item 
	\begin{itemize}
		\item Sceglie due numeri primi $ p $ e $ q $ e calcola $ N=p\times q $.
		\item Calcola $ \phi(N)=(p-1)\times (q-1) $. 
		\item Trova $ 0<e<N $ in modo che $ MCD(e,\phi(N))=1 $ (cioè siano coprimi).\\ La coppia $ (e,N) $ costituisce la chiave pubblica e viene inviata a Bob.
		\item Alice poi crea la sua chiave privata $ \mathbf{(d,N)} $ trovando $ d $ tale che $ \boxed{e\times d\equiv 1 \mod N } $.
	\end{itemize}
	\item Bob ricevuto $ (e,N) $ calcola: \[ C=m^e \mod N \] e lo invia a Alice. Quest'ultima per decifrare il messaggio eseguirà:\[ m=C^d \mod N \]
\end{enumerate}
\end{frame}

\begin{frame}
\frametitle{Perchè RSA funziona?}
Sappiamo che:
  $  C^d \mod N=m^{e\times d} \mod N$\\ inoltre  \[\boxed{ e\times d \equiv1 \mod \phi(N)} \] che equivale a:\[ e\times d \mod \phi(N) =1 \]
allora posso scrivere: \[ e\times d=k(\phi(N))+1=k(p-1)(q-1)+1 \] e quindi:
\[ m^{e\times d} \mod N=m^{k(p-1)(q-1)+1} 	\mod N=m^{k(p-1)(q-1)}\times m 	\mod N \]
 Se dimostriamo che $\boxed{ m^{k(p-1)(q-1)} 	\mod N=1 }$ otteniamo la tesi.  
\end{frame}
\begin{frame}
Poiché \[m^{k(p-1)(q-1)} 	\mod q=(m^{\mathbf{q-1}}) ^{k(p-1)} 	\mod q  \]
e per il Teorema piccolo Teorema di Fermat: \[\boxed{ m^{q-1} \equiv 1 \mod q} \]
si ha allora: \[ m^{k(p-1)(q-1)} 	\mod q=1^{k(p-1)} 	\mod q=1 \] analogamente:
\[m^{k(p-1)(q-1)} 	\mod p=[m^{p-1}]^{k(q-1)} 	\mod p=1^{k(q-1)} \mod p=1  \]

\end{frame}
\begin{frame}
Dai due risultati: \[  p m^{k(p-1)(q-1)} 	\mod q=1\]  \[  m^{k(p-1)(q-1)} 	\mod p=1  \]
Applicando il \textit{teorema cinese del resto} posso scrivere:

\[ m^{k(p-1)(q-1)} 	\mod (q\times p)= m^{k(p-1)(q-1)} 	\mod N=1 \]  
\end{frame}

\end{document}
